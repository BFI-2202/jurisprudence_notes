\documentclass{article}

\usepackage[T2A]{fontenc}
\usepackage[utf8]{inputenc}
\usepackage[russian]{babel}

\usepackage{tabularx}
\usepackage{amsmath}
\usepackage{pgfplots}
\usepackage{geometry}
\usepackage{multicol}
\geometry{
    left=1cm,right=1cm,top=2cm,bottom=2cm
}
\newcommand*\diff{\mathop{}\!\mathrm{d}}

\newtheorem{definition}{Определение}
\newtheorem{theorem}{Теорема}

\DeclareMathOperator{\sign}{sign}

\usepackage{hyperref}
\hypersetup{
    colorlinks, citecolor=black, filecolor=black, linkcolor=black, urlcolor=black
}

\title{Правоведение}
\author{Лисид Лаконский}
\date{Февраль 2023}

\begin{document}
\raggedright

\maketitle

\tableofcontents
\pagebreak

\section{Лекция — 09.02.2024}

\subsection{Правопонимание и право}

\begin{definition}
    \textbf{Естественное право} — доктрина в философии права и юриспруденции, признающая наличие у людей неотъемлемых прав, принадлежащих им от рождения.
\end{definition}

\begin{definition}
    \textbf{Позитивное право} — система общеобязательных норм, формализованных государством.
\end{definition}

Право обладает следующими \textbf{признаками}:

\begin{enumerate}
    \item \textbf{Системность и упорядоченность}
    \item \textbf{Нормативность}
    \item \textbf{Императивный властный характер}
    \item \textbf{Общеобязательность и общедоступность}
    \item \textbf{Формальная определенность}
    \item \textbf{Обладание регулятивным характером}
    \item \textbf{Всесторонняя обеспеченность и гарантированность}
\end{enumerate}

Право обладает \textbf{принципами}, подразделяющиеся на следующие:

\begin{enumerate}
    \item \textbf{Общие}:
    \begin{enumerate}
        \item Принцип справедливости
        \item Принцип законности
    \end{enumerate}
    \item \textbf{Межотраслевые}
    \item \textbf{Отраслевые}
\end{enumerate}

\subsection{Право и другие социальные нормы}

Отличие правовых норм от неправовых:

\begin{enumerate}
    \item \textbf{Характер отношений}, которые они регулируют: правовые нормы закрепляют основные правовые отношения между государством и гражданами; неправовые нормы регулируют межличностные отношения и прочее.
    \item \textbf{Порядок и способ установления}: неправовые нормы складываются сами собой, от поколения к поколению, иногда стихийно; правовые нормы издаются государством и содержатся в законодательных и иных правовых актах.
    \item \textbf{По форме и средствам обеспечения}: правовые нормы предусматривают государственные санкции и прочее.
    \item \textbf{Характер и степень определенности мер воздействия}
\end{enumerate}

\subsection{Источники (формы) права}

\begin{enumerate}
    \item \textbf{Правовые обычаи}
    \item \textbf{Нормативно правовые акты} государственных органов — выраженные в письменной форме решения компетентных государственных органов, в которых содержатся правовые нормы; законы, декреты, приказы министров и так далее.
    \item \textbf{Правовые договоры}
    \item \textbf{Прецеденты}
\end{enumerate}

\subsection{Иерархия нормативно правовых актов}

\begin{enumerate}
    \item Конституция Российской Федерации
    \item Международные акты
    \item Федеральный закон
    \item Указ президента Российской Федерации
    \item Постановление правительства
    \item Приказ министерства
    \item Приказ государственных служб
    \item Областной либо муниципальный закон
\end{enumerate}

\begin{definition}
    \textbf{Закон} — нормативно правовой акт, принимаемый высшим представительным органом государства в законодательном порядке, обладающий высшей юридической силой и регулирующий наиболее важные общественные отношения.
\end{definition}

\subsection{Классификация законов}

По значимости:

\begin{enumerate}
    \item \textbf{Конституционные} — законы, с помощью которых вводятся дополнения в Конституцию или законы, необходимость издания которых обуславливается конституцией.
    \item \textbf{Обыкновенные} (текущие) — все остальные законы
\end{enumerate}

В зависимости от территории распространения:

\begin{enumerate}
    \item \textbf{Федеральные}
    \item Принятые на уровне определенных субъектов Российской Федерации
\end{enumerate}

\subsection{Система права}

\begin{definition}
    \textbf{Системой законодательства} называется совокупность действующих в том или ином обществе нормативно правовых актов
\end{definition}

\begin{definition}
    \textbf{Правовой системой} (это не то же самое, что система права) называется правовая структура страны; правовая организация всего общества. К ней также относится правовая идеология, правосознание, правовая культура, правовая практика.
\end{definition}

\textbf{Структурными элементами} системы права являются \textbf{отрасли}, \textbf{институты} и \textbf{нормы}. \textbf{Нормы} являются первичными элементами системы права. Наиболее крупными ее структурными элементами являются \textbf{отрасли}.

\begin{definition}
    \textbf{Отрасль права} — совокупность относительно обособленных автономных юридических норм, регулирующих определенную область, сферу общественных отношений.
\end{definition}

\begin{definition}
    \textbf{Институты права} представляют собой обособленные группы взаимосвязанных юридических норм, регулирующих определенные разновидности общественных отношений. Существуют внутри \textbf{отраслей}
\end{definition}

\end{document}