\documentclass{article}

\usepackage[T2A]{fontenc}
\usepackage[utf8]{inputenc}
\usepackage[russian]{babel}

\usepackage{tabularx}
\usepackage{amsmath}
\usepackage{pgfplots}
\usepackage{geometry}
\usepackage{multicol}
\geometry{
    left=1cm,right=1cm,top=2cm,bottom=2cm
}
\newcommand*\diff{\mathop{}\!\mathrm{d}}

\newtheorem{definition}{Определение}
\newtheorem{theorem}{Теорема}

\DeclareMathOperator{\sign}{sign}

\usepackage{hyperref}
\hypersetup{
    colorlinks, citecolor=black, filecolor=black, linkcolor=black, urlcolor=black
}

\title{Правоведение}
\author{Лисид Лаконский}
\date{Февраль 2023}

\begin{document}
\raggedright

\maketitle

\tableofcontents
\pagebreak

\section{Практическое занятие — 03.02.2024}

\subsection{Знакомство}

\textbf{Преподаватель (ФИО)}: Лопаткина Александра Сергеевна, старший преподаватель, судья (Мосгорсуд) в отставке. \textbf{Для того, чтобы получить зачёт автоматом}, необходимо принимать активное участие в практических занятиях. Практические занятия лучше всего не пропускать. Один раз можно, систематически — недопустимо. Так же как и вообще пары лучше не прогуливать.

\end{document}